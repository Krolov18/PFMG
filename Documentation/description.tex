%! Author = Korantin
%! Date = 21/03/2022

% Preamble
\documentclass[12pt,a4paper,twoside]{article}

% Packages
\usepackage{amsmath}
\usepackage{fancyhdr}
\usepackage[T1]{fontenc}
\usepackage[utf8]{inputenc}
\usepackage{natbib}

\pagestyle{fancy}



\title{\vspace{-3\baselineskip}Construction du Kalabatiste}\label{toto}
\author{Korantin Lévêque}
\date{\today}

% Document
\begin{document}

\chead{\ref{Kalabatiste}}
\cfoot{Kalabatiste}


\maketitle
\tableofcontents

\section{Introduction}
L'objectif de ce document est renseigner au lecteur le cheminement quant à la réécriture de ce projet.
Le code du kalaba étant devenu complexe, l'instigateur a voulu ajouté une nouvelle fonctionnalité et n'a pu le faire.
Ma proposition consiste à appliquer plusieurs "design patterns" comme l'abstract factory.
C'est quoi le kalaba ?
Le kalaba est une grammaire lexicale permettant de reconnaitre, générer et traduire une langue source
vers une langue cible. Il faut peser tous les termes employés ici. La langue source est une langue inventée,
la langue cible, du français.
A quoi ça sert ?
A qui ça sert ?
\section{Code GB}
\subsection{Architecture}
Le Kalaba est un cas d'uage à l'implémentation de Paradigm Function Morphology de Stump.
\subsection{Usage des classes}
\subsection{Non usage des tests unitaires}


\section{Architecture et tests unitaires}
Il a fallu refondre l'architecture pour clarifier les structures de données qu'on allait manipuler.
En élaborant cette nouvelle architecture, les tests unitaires sont venus de pair.
\subsection{Vision fonctionnelle}
L'usage des classes ne e semblait pas nécessaire. J'ai ainsi choisi de surchargé des fonctions quand cela était nécessaire
plutôt que de sortir l'artillerie lourde avec les classes. Chaque module (fichier) contient une fonction
avec ses déclarations (fichiers .pyi) et ses implémentations (fichier .py).
\subsection{Le lexique}
On entend par lexique ici, une structure regroupant les lexème et les réalisations de ces lexèmes, soit, les formes.
Lexique est composé de deux piliers principaux : La glose du langage et les blocs de ce langage.

La glose répertorie tous les couples attributs/valeurs dont le langage aura besoin pour son bon fonctionnement.
Il est répertorié par catégorie.

Les blocs, aussi répertoriés par catégorie déclarent les règles de réalisations des lexèmes.
Un bloc relie un sigma (ensemble de traits/valeurs licite par la catégorie qui le recouvre) et une règle.
La règle est une règle de construction lexicale. Elle va ainsi permettre d'accumuler les différentes réalisations
décrites par le sigma.

\subsection{La syntaxe}
La syntaxe s'effectue à l'aide d'une Feature Contextual-Free Grammar (FCFG). Les catégories syntaxiques servants
de non-terminaux et les sigmas des features.
Le concepteur du langage aura donc uniquement à déclarer les règles non-lexicales. Un formalisme est mis
à sa disposition pour éviter le maximum de redondance.
\subsection{La traduction}
La traduction est l'étape la plus complexe car elle s'étale sur le lexique et sur la syntaxe.
\subsection{Vers du \LaTeX}


\section{HPSG via NLTK}
\subsection{PLY: Interface TDL avec FeatStruct}
\subsection{Du typé non-typé}
\subsection{Parsing et HPSG}

\end{document}
